

\documentclass[12pt]{article}
\usepackage[T1]{fontenc}
\usepackage[utf8]{inputenc}
\usepackage{amsmath}
\usepackage{microtype}
\usepackage{listings}
\setlength{\parindent}{0pt}
\usepackage{fancyvrb}
\usepackage{enumerate}
\usepackage{array}
\usepackage[breaklinks=true,linktocpage,hidelinks]{hyperref}
\usepackage[letterpaper]{geometry}
\usepackage{url}
\usepackage{graphicx}
\usepackage{fullpage}
\usepackage{caption}
\captionsetup{labelfont={bf,small,sf}, textfont={small, sf}}
\usepackage[sc]{mathpazo}

\usepackage{pgfplots}
\usepackage{pgfplotstable}
\usepackage{tikz}

\usepackage{fancyhdr}
\usepackage{fancybox}
\usepackage{multicol}
\usepackage{xcolor}
\usepackage{adjustbox}
\usepackage{wrapfig}



\pgfplotsset{compat=newest}
\usetikzlibrary{shapes,backgrounds,arrows}
\usepgfplotslibrary{external} 

\definecolor{brewcol1}{RGB}{166,206,227}
\definecolor{brewcol2}{RGB}{31,120,180}
\definecolor{brewcol3}{RGB}{178,223,138}
\definecolor{brewcol4}{RGB}{51,160,44}
\definecolor{brewcol5}{RGB}{251,154,153}
\definecolor{brewcol6}{RGB}{227,26,28}
\definecolor{brewcol7}{RGB}{237,179,1}
\definecolor{brewcol8}{RGB}{202,178,214}
\definecolor{brewcol9}{RGB}{206,27,1}

\definecolor{brewforest1}{RGB}{65,171,93}
\definecolor{brewforest2}{RGB}{161,217,155}
\definecolor{brewforest3}{RGB}{49,163,84}

\geometry{hmargin=1.87cm, vmargin=1.87cm}
\bibliographystyle{siam}

\DeclareTextFontCommand{\helvetica}{\fontfamily{phv}\selectfont\small}


\begin{document}

\clearpage\thispagestyle{empty}
\begin{center}
\textbf{Difficult transition for sugar maple in Boreal forest under climate change? \\
Impact of alternative stable states on Sugar maple migration.}
\vskip 2em
Research proposal
\vskip 1em
Master in Wildlife management
\vfill
By
\vfill
Steve Vissault 
\vfill 
For
\vfill
\textbf{Richard Cloutier}, Pr.\\
Director of the program committee
\vskip 2em
\textbf{Dominique Arsenault}, Pr.\\
President of the jury
\vskip 2em
\textbf{Matt Talluto}, PhD\\
Research Co-director
\vskip 2em
\textbf{Dominique Gravel}, Pr.\\
Research Director
\vfill
\vfill
Université du Québec à Rimouski\\
\today

\end{center}

\newpage
\setcounter{page}{1}

\section{Introduction.}

\textbf{Context.}  The boreal region is warming twice as fast as the global
average and will inevitably alter species composition in boreal forest
\cite{Scheffer2012,Hughes2000}.  Sugar maple is one of those species expected
to migrate northward towards it's nordic temperate forest limits
\cite{McKENNEY2007,Goldblum2005}. Predict shifts in the repartition of sugar
maple under climate change is an important challenge whereas this species is
highly coveted by wood and maple syrup producers, two main economic sectors in
Quebec. Indeed, Sugar maple is a widespread and abundant tree in north-eastern
North America and one of the most representative species of northern temperate
forests \cite{Graignic2013,Messaoud2007,Kellman2004}. This northward migration
will result in increasing the surface of the ecotone between the boreal and
temperate forest of Quebec. Nevertheless, the expansion of sugar maple
distribution could be difficult and explain by the fact than microclimatic
conditions found in boreal forests are different from those present in
temperate forest. Colder temperatures from shading and excess soil moisture
due to snow melt cause litter to be more acidic and fibrous during the spring.
Therefore, even if the regional climate conditions are favorable
\cite{Kellman2004}, the microbiota conditions found in the boreal forest could
affect the establishment of sugar maple
\cite{Kellman2004,Moore2008,DeFrenne2013}. In this case, the sugar maple could
be unable to migrate in boreal forests as a result of negative soil feedback.
This fact could increase the tension between the boreal forest and the nordic
temperate forest and generating abrupt changes in the species composition of
this ecotone.\\

\textbf{Theorical framework.}  Many ecotone studies and modeling efforts on
transition between forest to non-forest ecosystems (e. g. Boreal - Toundra)
\cite{Scheffer2012,Scheffer2001,Hirota2011} but little attention has been
given to evaluate the transitionnal dynamics of forest-forest ecotone
\cite{Goldblum2010,Graignic2013}. At regional scale, most transition between
differents ecosystems appear often gradually over the time
\cite{Scheffer2001,scheffer2009critical}. However, system can \\

\textbf{Natural system.}  We used maple sugar basa areal in function of black
spruce, white spruce and balsam fir basal area to compute the relative
abundance of sugar maple. Given the above statements, we used mainly two
climatic variables (annual mean temperature and annual precipitation) to
identify the alternative stable states present in the boreal-temperate
ecotone. The relationship was performed on climatic variables and sugar maple
relative abundance using kernel density plot function. We obtained the
probability of observed a sample plot as function of sugar maple relative
abundance and mean annual temperature (Figure \ref{fig1}). In both case,
alternative stable states are presents whereas the function graphed has a
bimodal distribution. At low precipitation or temperature conditions, the
probability of observing a parcel sampled without sugar maple is higher than
intermediate climatic conditions where the multimodal distribution appear. The
density function suggest a double hysterisis of sugar maple states in response
of intermediate climatic condition: one state wherein sugar maple dominates
and an another in which sugar maple is absent. \\

%% IMPORTANT:: Structurer les points

\begin{figure}[ht]
	\begin{center}
	\includegraphics[width=\textwidth]{fig/window_temp.pdf}
	\end{center}
	\caption{Kernel density estimate given the relative abundace of sugar maple. \textbf{Describe the procedure}}
	\label{fig1}
\end{figure}

This project aim to develop a state and transition model (STM) between the
boreal and temperate nordic forest in order to investigate the migration rate
of sugar maple under different climate change scenarios. To assess this
objective, we will use the alternative stable states theory as a framework.

%% Plan
% Theorical framework. 
% Applied this framework to the sugar maple natural system


%(General) finir avec l'objectifs


\section{Objectives.} 

This project aims to determine whether alternative stable states are present
in the temperate-boreal forest ecotone and if so, look at the impact of plant-
soil and disturbances feedback on the alternatives stables states. To assess
this main objective, we will (O1) generate a transitionnal model between the
temperate and the boreal forest; (O2) study the equilibrium states based on
the model; (O3) investigate the spatial structure of the transitonnal zone;
and finaly (O4) run simulations based on different climate change scenarios.
\\

\section{Methods.}

\textbf{Models.}

\textbf{Equilibrium study.}

\textbf{Paramerization.} 

\textbf{Validation.}


\newpage
\bibliography{/home/steve/Dropbox/Bibtex/Devis}
\end{document}
