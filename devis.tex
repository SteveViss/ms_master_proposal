
\documentclass[12pt]{article}
\usepackage[T1]{fontenc}
\usepackage[utf8]{inputenc}
\usepackage{amsmath}
\usepackage{microtype}
\usepackage{listings}
\setlength{\parindent}{0pt}
\usepackage{fancyvrb}
\usepackage{enumerate}
\usepackage{array}
\usepackage[breaklinks=true,linktocpage,hidelinks]{hyperref}
\usepackage[letterpaper]{geometry}
\usepackage{url}
\usepackage{graphicx}
\usepackage{fullpage}
\usepackage{caption}
\captionsetup{labelfont={bf,small,sf}, textfont={small, sf}}
\usepackage[sc]{mathpazo}

\usepackage{pgfplots}
\usepackage{pgfplotstable}
\usepackage{tikz}

\usepackage{fancyhdr}
\usepackage{fancybox}
\usepackage{multicol}
\usepackage{xcolor}
\usepackage{adjustbox}
\usepackage{wrapfig}



\pgfplotsset{compat=newest}
\usetikzlibrary{shapes,backgrounds,arrows}
\usepgfplotslibrary{external} 

\definecolor{brewcol1}{RGB}{166,206,227}
\definecolor{brewcol2}{RGB}{31,120,180}
\definecolor{brewcol3}{RGB}{178,223,138}
\definecolor{brewcol4}{RGB}{51,160,44}
\definecolor{brewcol5}{RGB}{251,154,153}
\definecolor{brewcol6}{RGB}{227,26,28}
\definecolor{brewcol7}{RGB}{237,179,1}
\definecolor{brewcol8}{RGB}{202,178,214}
\definecolor{brewcol9}{RGB}{206,27,1}

\definecolor{brewforest1}{RGB}{65,171,93}
\definecolor{brewforest2}{RGB}{161,217,155}
\definecolor{brewforest3}{RGB}{49,163,84}

\geometry{hmargin=1.87cm, vmargin=1.87cm}
\bibliographystyle{siam}

\DeclareTextFontCommand{\helvetica}{\fontfamily{phv}\selectfont\small}


\begin{document}

\clearpage\thispagestyle{empty}
\begin{center}
\textbf{Difficult transition for sugar maple in Boreal forest under climate change? \\
Impact of alternative stable states on Sugar maple migration.}
\vskip 2em
Research proposal
\vskip 1em
Master in Wildlife management
\vfill
By
\vfill
Steve Vissault 
\vfill 
For
\vfill
\textbf{Richard Cloutier}, Pr.\\
Director of the program committee
\vskip 2em
\textbf{Dominique Arsenault}, Pr.\\
President of the jury
\vskip 2em
\textbf{Matt Talluto}, PhD\\
Research Co-director
\vskip 2em
\textbf{Dominique Gravel}, Pr.\\
Research Director
\vfill
\vfill
Université du Québec à Rimouski\\
\today

\end{center}


\newpage
\setcounter{page}{1}
\textbf{Introduction.} Many ecotone studies and modeling efforts on transition between forest to non-forest ecosystems \cite{Scheffer2012,Scheffer2001,Hirota2011} but little attention has been given to evaluate the transitionnal dynamics of forest-forest ecotone \cite{Goldblum2010,Graignic2013}. Ecotones most present in Quebec consists of a transitionnal area with intermixed fragments of boreal forest and northern temperate forest\cite{Goldblum2010}. This area is hypothesized to shift in the coming decades due to the climate change \cite{Scheffer2012}. Indeed, this zone is of particulary concern as it contains species pushed to their physiological limits, generating a zone sensitive to environmental fluctuations such as those from a climatic change \cite{Messaoud2007,Goldblum2010}. The boreal region is warming twice as fast as the global average and will inevitably alter interaction and composition in the community \cite{Scheffer2012,Hughes2000}. Hence, we expected  this area is undergoing significant natural pressures (e.g. fire regime) increasing in frequency and severity in the coming decades and inducing profound changes due to climate change. Many species of the boreal-temperate forest ecotone are expected to expand or reduce their distribution \cite{Graignic2013,Goldblum2005,Hughes2000} modifying the composition and the dynamic of this ecotone.\\

% Review the transitional sentences to the species Sugar maple (central in this topics).
% Despite macroclimatic conditions favorable to the expansion of sugar maple   
 Sugar maple is one of the most representative species of northern temperate forests \cite{Graignic2013,Messaoud2007,Kellman2004}. Currently some patches have been established in the boreal landscape  and in the coming decades this species is expected to continue to extend its range \cite{Graignic2013,Goldblum2005,Woodall2009}. This northward migration will result in increasing the surface of the ecotone between the boreal and temperate forest of Quebec. The expansion of sugar maple distribution could be difficult and explain by the fact than microclimatic conditions found in boreal forests are different from those present in temperate forest. Colder temperatures from shading and excess soil moisture due to snow melt cause litter to be more acidic and fibrous during the spring. Therefore, even if the regional climate conditions are favorable \cite{Kellman2004}, the edaphic conditions found in the boreal forest could affect the establishment of sugar maple \cite{Kellman2004,Moore2008}. In this case, the sugar maple could be unable to migrate in boreal forests as a result of climate change. This hypothesis correspond with the observations of Zhu \emph{et al} (2012) who revealed than only 4.3\% of migrations are consistent with expansion\cite{Zhu2012}.\\

%%%% Voir pour détailler, 

Our hypothesis proposes the presence of alternative stable states in the transition zone between the temperate and boreal forest as a result of the progression difficulty. Under pressure from the same climatic conditions we would expect to find in boreal-temperate forests ecotone, the alternative stable states consisting of forest patches dominated by balsam fir and an another by sugar maple at same environnemental conditions and inside the contraction zone between boreal and temperate forests. \\

\textbf{Objective.} This thesis aims to determine whether alternative stable states are present in the temperate-boreal forest ecotone and if so, look at the impact of plant-soil and disturbances feedback on the alternatives stables states. To assess this main objective, we will (O1) generate a transitionnal model between the temperate and the boreal forest; (ii) study the equilibrium states based on the model; (iii) investigate the spatial structure of the transitonnal zone; and finaly (vi) run simulations based on different climate change scenarios. 

\textbf{Methods.}



	\begin{figure}[!ht]
\begin{center}
	
				\tikzstyle{noeud}=[circle,
				                  thick,
				                  minimum size = 1.5cm,
				                  inner sep =5pt,
				                  draw=brewforest3,
				                  fill=brewforest1]
				\tikzstyle{noeud2}=[circle,
				                  thick,
				                  minimum size = 1.5cm,
				                  inner sep =5pt,
				                  draw=brewforest3,
				                  fill=brewforest2]
				\tikzstyle{noeud3}=[circle,
				                  thick,
				                  minimum size = 1.5cm,
				                  inner sep =5pt,
				                  draw=brewforest3,
				                  fill=brewforest3]
	
				\begin{tikzpicture}[->,>=stealth',auto,scale=0.8]
				      \node [circle,noeud2] (M) at (0,0) {\textbf{M}};
				      \node [circle,noeud2]  (C) at (-5,5) {\textbf{C}};
				      \node [circle,noeud2] (D) at (5,5) {\textbf{D}};
				      \node [circle,noeud2] (T) at (0,10) {\textbf{T}};
	
						\draw[thick,-latex] (M) to[bend right=10] node[above,sloped] {$S_C$} (C);
						\draw[thick,-latex] (C) to[bend right=10] node[below,sloped] {$\beta_d \cdot (C+M)$} (M);
	
						\draw[thick,-latex] (D) to[bend right=10] node[above,sloped] {$\beta_c \cdot (D+M)$} (M);
						\draw[thick,-latex] (M) to[bend right=10] node[below,sloped] {$S_D$} (D);
	
						\draw[thick,-latex] (D) to[bend right=10] node[above,sloped] {$e$} (T);
						\draw[thick,-latex] (T) to[bend right=10] node[below,sloped] {$P_D$} (D);
	
						\draw[thick,-latex] (T) to[bend right=10] node[above,sloped] {$P_C$} (C);
						\draw[thick,-latex] (C) to[bend right=10] node[below,sloped] {$e$} (T);
	
						\draw[thick,-latex,transform canvas={xshift=0.8ex}] (T) to node[above,sloped,rotate=90,transform canvas={xshift=5ex}] {$P_D \cdot P_C$} (M);
						\draw[thick,-latex,transform canvas={xshift=-0.8ex}] (M) to node[above,sloped,rotate=-90,transform canvas={xshift=-3ex}] {$e$} (T);
				\end{tikzpicture}
	
			\caption{Conceptual transition model between forest stands deciduous ($D$), mixte ($M$) and coniferious ($C$). $E$ corresponds to a transitionnal state where a perturbation are occurred with a frequence of $e$. Parameters $\beta$ and $S$ are referred as the colonisation and the succession rates respectively. We defined the recovery rates ($P_C$ et $P_D$) as $P_C = \alpha_C \cdot (M+C) \cdot [1- \alpha_D \cdot (D +M)]$ and $P_D = \alpha_D \cdot (D+M) \cdot [1- \alpha_C \cdot (C +M)]$, to finaly get this equation $P_M = P_C \cdot P_D$. The parameter $\alpha$ mean the recovery rate after a patch has been disturbe.}

	\end{center}	
	\label{Model}
\end{figure}

\newpage

\helvetica{
\colorbox{brewcol1!40}{\parbox{\textwidth}{\textbf{\color{brewcol2}{Definition} of terms}
\begin{description}
	\item[\helvetica{Ecotone:}] Livingston (1903) defined an ecotone as a zone of tension between forest ecosystems \cite{Goldblum2010}. 
	\item[\helvetica{Alternative stable states:}] An alternative stable state is described as contrasting states in which a system may converge under the same external conditions \cite{scheffer2009critical}
\end{description}}}
}


\newpage
\bibliography{/home/steve/Dropbox/Bibtex/Devis}
\end{document}
