



\documentclass[12pt]{article}
\usepackage[T1]{fontenc}
\usepackage[utf8]{inputenc}
\usepackage{amsmath}
\usepackage{microtype}
\usepackage{listings}
\setlength{\parindent}{0pt}
\usepackage{fancyvrb}
\usepackage{enumerate}
\usepackage{array}
\usepackage[breaklinks=true,linktocpage,hidelinks]{hyperref}
\usepackage[letterpaper]{geometry}
\usepackage{url}
\usepackage{graphicx}
\usepackage{fullpage}
\usepackage{caption}
\captionsetup{labelfont={bf,small,sf}, textfont={small, sf}}
\usepackage[sc]{mathpazo}

\usepackage{pgfplots}
\usepackage{pgfplotstable}
\usepackage{tikz}

\usepackage{fancyhdr}
\usepackage{fancybox}
\usepackage{multicol}
\usepackage{xcolor}
\usepackage{adjustbox}
\usepackage{wrapfig}



\pgfplotsset{compat=newest}
\usetikzlibrary{shapes,backgrounds,arrows}
\usepgfplotslibrary{external} 

\definecolor{brewcol1}{RGB}{166,206,227}
\definecolor{brewcol2}{RGB}{31,120,180}
\definecolor{brewcol3}{RGB}{178,223,138}
\definecolor{brewcol4}{RGB}{51,160,44}
\definecolor{brewcol5}{RGB}{251,154,153}
\definecolor{brewcol6}{RGB}{227,26,28}
\definecolor{brewcol7}{RGB}{237,179,1}
\definecolor{brewcol8}{RGB}{202,178,214}
\definecolor{brewcol9}{RGB}{206,27,1}

\definecolor{brewforest1}{RGB}{65,171,93}
\definecolor{brewforest2}{RGB}{161,217,155}
\definecolor{brewforest3}{RGB}{49,163,84}

\geometry{hmargin=1.87cm, vmargin=1.87cm}
\bibliographystyle{siam}

\DeclareTextFontCommand{\helvetica}{\fontfamily{phv}\selectfont\small}


\begin{document}

\clearpage\thispagestyle{empty}
\begin{center}
\textbf{Difficult transition for sugar maple in Boreal forest under climate change? \\
Impact of alternative stable states on Sugar maple migration.}
\vskip 2em
Research proposal
\vskip 1em
Master in Wildlife management
\vfill
By
\vfill
Steve Vissault 
\vfill 
For
\vfill
\textbf{Richard Cloutier}, Pr.\\
Director of the program committee
\vskip 2em
\textbf{Dominique Arsenault}, Pr.\\
President of the jury
\vskip 2em
\textbf{Matt Talluto}, PhD\\
Research Co-director
\vskip 2em
\textbf{Dominique Gravel}, Pr.\\
Research Director
\vfill
\vfill
Université du Québec à Rimouski\\
\today

\end{center}

\newpage
\setcounter{page}{1}

\section{Introduction.}

\textbf{Context.} The boreal region is warming twice as fast as the global average and will inevitably alter species composition in boreal forest \cite{Scheffer2012,Hughes2000}.  Sugar maple is one of those species expected to migrate northward towards it's northern limits \cite{McKENNEY2007,Goldblum2005}. Predict shifts in the repartition of sugar maple under climate change is an important challenge whereas this species is highly coveted by wood and maple syrup producers, two main economic sectors in Quebec. Indeed, Sugar maple is a widespread and abundant tree in north-eastern North America and one of the most representative species of northern temperate forests \cite{Graignic2013,Messaoud2007,Kellman2004}. This northward migration will result in increasing the surface of the ecotone between the boreal and temperate forest of Quebec. Many ecotone studies and modeling efforts on transition between forest to non-forest ecosystems \cite{Scheffer2012,Scheffer2001,Hirota2011} but little attention has been given to evaluate the transitionnal dynamics of forest-forest ecotone \cite{Goldblum2010,Graignic2013}. This project aim to develop a transitionnal model between the boreal and temperate nordic forest in order to investigate the migration rate of sugar maple under climate change. \\

\textbf{Theorical framework.}   %(General) finir avec l'objectifs


\section{Objectives.}

 This project aims to determine whether alternative stable states are present in the temperate-boreal forest ecotone and if so, look at the impact of plant-soil and disturbances feedback on the alternatives stables states. To assess this main objective, we will (O1) generate a transitionnal model between the temperate and the boreal forest; (O2) study the equilibrium states based on the model; (O3) investigate the spatial structure of the transitonnal zone; and finaly (O4) run simulations based on different climate change scenarios. \\

\section{Methods.}


\newpage
\bibliography{/home/steve/Dropbox/Bibtex/Devis}
\end{document}
