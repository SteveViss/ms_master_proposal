


Many ecotone studies and modeling efforts on transition between forest to non-forest ecosystems \cite{Scheffer2012,Scheffer2001,Hirota2011} but little attention has been given to evaluate the transitionnal dynamics of forest-forest ecotone \cite{Goldblum2010,Graignic2013}. Ecotones most present in Quebec consists of a transitionnal area with intermixed fragments of boreal forest and northern temperate forest\cite{Goldblum2010}. This area is hypothesized to shift in the coming decades due to the climate change \cite{Scheffer2012}. Indeed, this zone is of particulary concern as it contains species pushed to their physiological limits, generating a zone sensitive to environmental fluctuations such as those from a climatic change \cite{Messaoud2007,Goldblum2010}. The boreal region is warming twice as fast as the global average and will inevitably alter interaction and composition in the community \cite{Scheffer2012,Hughes2000}. Many species of the boreal-temperate forest ecotone are expected to expand or reduce their distribution \cite{Graignic2013,Goldblum2005,Hughes2000} modifying the composition and the dynamic of this ecotone.\\


Most transition in different ecosystem appear often gradually over the time \cite{Scheffer2001,scheffer2009critical}. However, when a disturbance undergoing in systems can responds in two ways depending   \\

 Lorsqu'une perturbation intervient sur un écosysteme, deux types de réponses possibles: 1- Réponse 'smooth' ou alors on 'catastrophic shift'. La perturbation fait oscillié le système atours de son point d'équilibre. Si le système atteint le point de bifurcation alors on observe un changement d'état (transition vers un état stable alternatif). Si le point de bifurcation n'est pas atteint alors le retour au point d'équilibre se fait en fonction de la capacité de résilience du systeme (la pente).  

Our hypothesis proposes the presence of alternative stable states in the transition zone between the temperate and boreal forest as a result of the progression difficulty. Under pressure from the same climatic conditions we would expect to find in boreal-temperate forests ecotone, the alternative stable states consisting of forest patches dominated by balsam fir and an another by sugar maple at same environnemental conditions\\ 

Sugar maple is one of the most representative species of northern temperate forests \cite{Graignic2013,Messaoud2007,Kellman2004}. Currently some patches have been established in the boreal landscape  and in the coming decades this species is expected to continue to extend its range \cite{Graignic2013,Goldblum2005,Woodall2009}. This northward migration will result in increasing the surface of the ecotone between the boreal and temperate forest of Quebec. The expansion of sugar maple distribution could be difficult and explain by the fact than microclimatic conditions found in boreal forests are different from those present in temperate forest. Colder temperatures from shading and excess soil moisture due to snow melt cause litter to be more acidic and fibrous during the spring. Therefore, even if the regional climate conditions are favorable \cite{Kellman2004}, the microbiota conditions found in the boreal forest could affect the establishment of sugar maple \cite{Kellman2004,Moore2008,DeFrenne2013}. In this case, the sugar maple could be unable to migrate in boreal forests as a result of climate change. This hypothesis correspond with the observations of Zhu \emph{et al} (2012) who revealed than only 4.3\% of migrations are consistent with expansion\cite{Zhu2012}.\\

%%%% Voir pour détailler, 